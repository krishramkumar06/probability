%ch3.tex
\chapter{Combinatorics}
\section{Permutations}

	\begin{oddenumerate}
		\item $ 4! $
		\item $ 2^{32} $
		\item $ 9; 6 $
		\item $ \dfrac{5!}{5^5} $
		\item lower: 40315.88809741376
		\\ middle: 40320
		\\ upper: 40320.21778522546
		\\ Inequality holds for n = 8
		\item the probability a certain candidate $ k $ is hired is $ (3/n^2 - 2/n^3) $. The probability a candidate is hired at all is $ (2/n) - 1/n^2 $
		\item (a) $ 10^3 * 26^3 $, (b) $  \binom{6}{3} * 10^7 * 26^3$
		\item $  (\dfrac{2}{3})^n - 2*(\dfrac{1}{3})^n  $
		\item The probability is 1 if n is greater than 12. Otherwise, it is \[ 1 - \dfrac{11!}{12^{n-1}(12-n)!} \]
		\item for n =  10 the real d is  5
		\\ The estimate yields  3.723297411059034  and the variance is  1.276702588940966
		\\ for n =  31 the real d is  7
		\\ The estimate yields  6.555541563800554  and the variance is  0.44445843619944636
		\\ for n =  100 the real d is  13
		\\ The estimate yields  11.774100225154747  and the variance is  1.225899774845253
		\\ for n =  365 the real d is  23
		\\ The estimate yields  22.49438689559598  and the variance is  0.50561310440402
		\item for permutations of size 9:
		\\ 0 fixed points: 133496
		\\ 1 fixed points: 133497
		\\ 2 fixed points: 66744
		\\ 3 fixed points: 22260
		\\ 4 fixed points: 5544
		\\ 5 fixed points: 1134
		\\ 6 fixed points: 168
		\\ 7 fixed points: 36
		\\ 8 fixed points: 0
		\\ 9 fixed points: 1
		\\ 
		\\ Derangements:  0.36787918871252206
		\\ 1 fix:  0.36788194444444444
		\\ Conjecture, both of them approach $ 1/e * n!$
		\item theres a 1/2 chance that 2nd place is on the left (meaning that nobody better will make us pass during the second half) and a 1/2 chance that 1 is on the right, meaning we will succeed. 
		code for 24 final ratio:  0.40245
	\end{oddenumerate}

\section{Combinations}

	\begin{oddenumerate}
	\item (a) $ \binom{6}{3} = 20 $; (b) b(5,.2,4) = .0064; (c) 21; (d) 1; (e) .0256; (f) 15; (g) 10; (h) .04668 
	\item 36
	\item The probability that, in 100 tosses of a fair coin, the number of heads that turns up lies between
	\\ (1) 35 and 65 0.9987
	\\ (2) 40 and 60 0.9653
	\\ (3) 45 and 55 0.7387
	\item $ b(n,p,j) = \binom{n}{j} * p^j *q^{n-j} = \frac{p}{q} * \frac{n-j+1}{j} * \binom{n}{j-1} * p^{j-1}*q^{n-j+1}
	= \\ \frac{p}{q} * \frac{n-j+1}{j} * b(n,p,j-1)$. We want the fraction$  \dfrac{n-j+1}{j}  $ to be greater than 1, this ends at $ j = \frac{n+1}{2} $, so that is the maximum value.
	\item $ \binom{15}{7} $
	\item The closest an owner can get to $ 95 \% $, then they can have $ 7 $ apple and $ 10 $ blueberry. If the owner cares about being only confident over 95 and wants to be efficient, have $ 8 $ of each type of pie.
	\item Without loss of generality, assume that $ k \leq n $. We want to show that $ \binom{2n}{k} $ is maximized when $ k = n $. 
	\begin{align*}
		\binom{2n}{k} &\leq \binom{2n}{k} * \frac{2n - k}{n} * \frac{2n - k - 1}{n - 1}*...*\frac{n+1}{k+1}
		\\ &= \binom{2n}{k} * \dfrac{k!}{n!} * \dfrac{(2n-k)!}{n!}
		\\ &= \binom{2n}{n}
	\end{align*}
	\item .343, .441, .189, .027
	\item$  \binom{8}{3} * \binom{5}{3} * \binom{2}{2} $, $ \binom{n}{a} * \binom{n-a}{b}  * \binom{n-a-b}{c}   = \dfrac{n!}{a!b!c!} $
	\item (a) $ \binom{4}{1} *\binom{13}{10}  $, (b) $ \binom{4}{3}  * 3 * \binom{13}{4}  * \binom{13}{3}  * \binom{13}{3} $, 
	(c) $ 4! * \binom{13}{4}  * \binom{13}{3}* \binom{13}{2} * \binom{13}{1}  $
	\item $ \binom{3}{2} * 2^5 - 3 $
	\item You need n-1 bars for n boxes, and having r stars means that you choose the r positions for the stars around the bars
	$ \implies \binom{n-1+r}{r}  $
	\item Case 1: indistinguishable passangers yields $ \dfrac{\binom{10}{6}}{\binom{15}{6}} \approx 4.2 \% $
	\\ Case 2: distinguishable passangers yields $ (10*9*8*7*6*5)/10^6 \approx 15.1 \% $
	\item With 100 participants, the critical value can be anywhere between 60 and 68
	\item We want to choose the value for $ p $ that maximizes the probability of getting $ m $ sucesses in $ n $ trials. This means, to maximize $ b(n,p,j) $ with $ p $ being what we can vary, we differentiate $ \binom{n}{m}*p^{m}(1-p)^{n-m} $ with respect to $ p $ and equate to zero. This yields $ \binom{n}{m} * p^{n-1} * (1-p)^{n-m-1}[p(m-n)+m(1-p)] = 0 \implies m - np = 0 \implies p = \dfrac{n}{m} $
	\item $ \sum_{i=2}^4 * b(4,0.99,i)  = 0.99999603$
	\item $ \dfrac{\binom{2n}{n}}{2^{2n}} \approx \dfrac{1}{\sqrt{\pi n}} $
	\\ for $ n = 20 , \binom{40}{20}/2^{40} = 0.12537068, \dfrac{1}{\sqrt{\pi * 20 }} = 0.1261566261$, and their difference is $ 0.00078 $
	\item To choose $ n $ balls from the urn, you can compute the ways irrespective of the ball's color ($ \binom{2n}{n} $), or you iterate $ i $ from 0 to $ n $ to have $ i $ red balls and $ n - i $ blue balls. Changing the denominator of the binomial coefficient yields the right hand side. This argument can be extended to produce Vandermonde's identity.

	\item 
	
	\begin{verbatim}
		[1]
		[1, 1]
		[1, 0, 1]
		[1, 1, 1, 1]
		[1, 0, 0, 0, 1]
		[1, 1, 0, 0, 1, 1]
		[1, 0, 1, 0, 1, 0, 1]
		[1, 1, 1, 1, 1, 1, 1, 1]
		[1, 0, 0, 0, 0, 0, 0, 0, 1]
		[1, 1, 0, 0, 0, 0, 0, 0, 1, 1]
		[1, 0, 1, 0, 0, 0, 0, 0, 1, 0, 1]
		[1, 1, 1, 1, 0, 0, 0, 0, 1, 1, 1, 1]
		[1, 0, 0, 0, 1, 0, 0, 0, 1, 0, 0, 0, 1]
		[1, 1, 0, 0, 1, 1, 0, 0, 1, 1, 0, 0, 1, 1]
		[1, 0, 1, 0, 1, 0, 1, 0, 1, 0, 1, 0, 1, 0, 1]
		[1, 1, 1, 1, 1, 1, 1, 1, 1, 1, 1, 1, 1, 1, 1, 1]
		[1, 0, 0, 0, 0, 0, 0, 0, 0, 0, 0, 0, 0, 0, 0, 0, 1]
	\end{verbatim}
	Because the $ 2^k $th row must have all zeroes besides its ends, this forces the previous row to have all ones.

	\item $ b(2n, .5,n) = \binom{2n}{n} * .5^{2n}  = \dfrac{2^n (n!) * odds}{\frac{evens}{2^n}} * \dfrac{1}{2^{2n}} = \dfrac{odds}{evens}$

	\end{oddenumerate}

\section{Card Shuffling}
%TODO when i feel like this, return to this section